\RequirePackage[l2tabu, orthodox]{nag}

\documentclass{jsarticle}
\usepackage[dvipdfmx]{graphicx}
\usepackage{otf}
\usepackage{comment}
\usepackage{ascmac}
\usepackage{amsmath, amssymb}
\usepackage{amsfonts}
\usepackage{bm}
\usepackage{mathtools}
\usepackage{empheq}
\usepackage{wrapfig}
\usepackage{multirow}
\usepackage{listings}
\lstset{
  language={Python},
  basicstyle={\ttfamily},
  identifierstyle={\small}, % style for non-keywords
  % commentstyle={\smallitshape}, ??
  keywordstyle={\small\bfseries},
  ndkeywordstyle={\small},
  stringstyle={\small	tfamily},
  frame={tb}, % ??
  breaklines=true, % ??
  columns=[l]{fullflexible},
  xrightmargin=0zw,
  xleftmargin=3zw,
  numbers=left, % line number[none, left, right]
  numberstyle={\scriptsize},
  stepnumber=1,
  numbersep=1zw,
  lineskip=-1ex
}
\renewcommand{\lstlistingname}{Code}

\title{}

\author{05215502 生物情報科学科 伊藤洸規}
\date{}
\begin{document}
\maketitle

\section*{}
この問題を解くには、指数法則と除算を利用します。まず、$\dfrac{10^{210}}{10^{10}+3}$を簡単に表現しましょう。

\begin{enumerate}
  \item $10^{210}$は、$10^{200} \times 10^{10}$と書くことができます。これにより、指数法則を利用できる形になります。
  \item 式を次のように分解しましょう。

  \[\dfrac{10^{210}}{10^{10}+3} = \dfrac{10^{200} \times 10^{10}}{10^{10}+3}\]

  \item ここで、分子と分母に$10^{10}$をかけることで、式をさらに簡単にできます。

  \[\dfrac{10^{200} \times 10^{10}}{10^{10}+3} = \dfrac{10^{200}}{\dfrac{10^{10}+3}{10^{10}}}\]

  \item 分母の式を簡単化します。

  \[\dfrac{10^{10}+3}{10^{10}} = 1 + \dfrac{3}{10^{10}}\]

  \item 式は次のようになります。

  \[\dfrac{10^{200}}{1+\dfrac{3}{10^{10}}}\]

  \item ここで、分母の1に対して$\dfrac{3}{10^{10}}$は非常に小さいため、無視できるほど小さいと考えることができます。そのため、式は次のように近似できます。

  \[\dfrac{10^{200}}{1} = 10^{200}\]

\end{enumerate}

この近似により、整数部分の桁数は200桁であることがわかります。

次に、1の位の数字を求めましょう。この部分は近似が難しいため、元の式を使用して求めます。

\begin{enumerate}
  \item $3^{21}=10460353203$が既知であることを利用し、元の式を次のように書き換えます。

  \[\dfrac{10^{210}}{10^{10}+3} = \dfrac{10^{210}}{10^{10}+3^{21}}\]

  \item ここで、分子は10の倍数であり、分母は1の位が3であることがわかります。したがって、除算の結果は1の位が7になることがわかります。これは、10の倍数を
  を3で割ったとき、余りが1になることから導かれます(例:10÷3=3...1、20÷3=6...2、30÷3=10...0)。
\end{enumerate}

以上の解法により、$\dfrac{10^{210}}{10^{10}+3}$の整数部分の桁数は200桁であり、1の位の数字は7です。これにより、問題の答えが求められました。


$\cos$
\end{document}